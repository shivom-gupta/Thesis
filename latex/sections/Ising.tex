\section[Mathematical Formulation of the 1D Ising
Model]{\texorpdfstring{Mathematical Formulation of the 1D
Ising\\Model}{Mathematical Formulation of the 1D Ising Model}}
\label{sec:ising1d}

The Ising Model, a theoretical construct, represents a lattice of sites, each of
which can exist in one of two states: -1 or +1. These states are denoted as
$\sigma_i$, where $i$ is the site index. For instance, $\sigma_i = -1$ indicates
that the $i$-th site is in the state -1.

\subsection{The Hamiltonian}
The Hamiltonian of the Ising Model includes two main components: the interaction
energy between nearest neighboring spins and the individual energy of each spin
due to an external magnetic field. Mathematically, it is represented as:
\begin{equation}
  \label{eq:hamiltonian}
  H = -J \sum_{\langle i,j \rangle} \sigma_i \sigma_j - h \sum_{i} \sigma_i
\end{equation}

Here, the first term sums the interactions of adjacent spins, denoted by
$\langle i,j \rangle$, indicating summation over nearest neighbors. The second
term sums the individual spin energies, with $h$ representing the external
magnetic field's strength. The coupling constant $J$ determines the interaction
strength between neighboring spins, being positive for ferromagnetic and
negative for antiferromagnetic interactions. This mathematical representation is
central to understanding the physical implications and behaviors modeled by the
Ising Model.

\subsection{The Partition Function}
The partition function $Z$ is a central concept in statistical mechanics. It
represents the sum of all possible states of a system, weighted by their
respective Boltzmann factors. For the Ising Model, the partition function is
given by:
\begin{equation}
  \label{eq:partition}
  Z = \sum_{\{\sigma_i\}} e^{-\beta H}
\end{equation}

Here, $\beta = \frac{1}{k_B T}$, where $k_B$ is the Boltzmann constant and $T$
is the temperature. The sum is over all possible states of the system, denoted
by $\{\sigma_i\}$. The Boltzmann factor $e^{-\beta H}$ is a measure of the
probability of a state occurring, with lower energy states being more probable.
The partition function is a central concept in statistical mechanics, as it
allows us to calculate the thermodynamic properties of a system.

\subsection{Exact Solution of the 1D Ising Model}
The 1D Ising Model can be solved exactly, as demonstrated by Ernst Ising in his
1924 thesis. This solution employs the transfer matrix method, where the
partition function is formulated as a product of matrices. The partition
function for the 1D Ising Model is defined as:
\begin{equation}
  \label{eq:partition1d}
  Z = \sum_{\{\sigma_i\}} e^{-\beta H} = \sum_{\{\sigma_i\}} \prod_{i=1}^{N-1} e^{-\beta H_i}
\end{equation}
where \( N \) is the number of sites on the lattice, and the Hamiltonian for the
interaction between the \( i \)-th and \( (i+1) \)-th site is:
\begin{equation}
  \label{eq:hamiltonian1d}
  H_i = -J \sigma_i \sigma_{i+1} - \frac{h}{2} (\sigma_i + \sigma_{i+1})
\end{equation}

The partition function can be expressed as a product of matrices using the
transfer matrix method. The transfer matrix \( T \) for each pair of adjacent
spins is given by:
\begin{equation}
  \label{eq:transfer1d}
  T = \begin{pmatrix}
    e^{\beta(J+h)} & e^{-\beta J} \\
    e^{-\beta J} & e^{\beta(J-h)}
  \end{pmatrix}
\end{equation}

The partition function in terms of the transfer matrix is:
\begin{equation}
  \label{eq:partition1dmat2}
  Z = \sum_{\{\sigma_i\}} T^{N-1} = \text{Tr}(T^N)
\end{equation}
where \( \text{Tr} \) denotes the trace of the matrix. This expression shows
that the partition function can be calculated by raising the transfer matrix to
the power of \( N \) and taking its trace, providing a complete solution to the
1D Ising Model.

To solve this, we diagonalize the matrix \( T \). The eigenvalues \( \lambda_1
\) and \( \lambda_2 \) of \( T \) are found by solving the characteristic
equation, which is the determinant of \( T - \lambda I \), where \( I \) is the
identity matrix. The characteristic equation is:

\begin{equation}
\text{det}(T - \lambda I) = \begin{vmatrix}
    e^{\beta(J+h)} - \lambda & e^{-\beta J} \\
    e^{-\beta J} & e^{\beta(J-h)} - \lambda
\end{vmatrix} = 0
\end{equation}

Solving this equation gives us the eigenvalues \( \lambda_1 \) and \( \lambda_2
\). The partition function \( Z \) can then be expressed as:

\begin{equation}
Z = \text{Tr}(T^N) = \lambda_1^N + \lambda_2^N
\end{equation}

This formulation of \( Z \) encapsulates the sum over all possible
configurations of the spins, weighted by their Boltzmann factor \( e^{-\beta H}
\). The eigenvalues, functions of the temperature \( \beta \), coupling constant
\( J \), and external magnetic field \( h \), determine the system's behavior.

In the thermodynamic limit, where \( N \) tends to infinity, the partition
function is dominated by the largest eigenvalue, as the contribution from the
smaller eigenvalue becomes negligible. This results in the final expression for
the partition function:

\begin{equation}
Z \approx \lambda_{\text{max}}^N
\label{eq:partitionmax}
\end{equation}

where \( \lambda_{\text{max}} \) is the larger of the two eigenvalues \(
\lambda_1 \) and \( \lambda_2 \). According to the Perron-Frobenius theorem, \(
\lambda_{\text{max}} \) is positive and real, affirming that the partition
function \( Z \) is positive and real, a necessary condition for a physical
system. This validates that the 1D Ising Model is not merely a mathematical
abstraction but represents a physically realizable system.

In the special case of \( h = 0 \), the eigenvalues simplify to:
\begin{equation}
 \lambda_1 = 2 \cosh(\beta J), \quad \lambda_2 = 2 \sinh(\beta J)
\end{equation}

These eigenvalues, especially under special conditions like \( h = 0 \), can be
computed analytically, providing valuable insights into the thermodynamic
properties of the system. This exact solution is a fundamental result in
statistical mechanics, demonstrating the efficacy of the transfer matrix method
in solving one-dimensional models.

\subsection{Thermodynamic Properties of the 1D Ising Model}

As discuessed in equation \ref{eq:partitionmax}, the partition function is given
by the largest eigenvalue \( \lambda_{\text{max}} \) in the thermodynamic limit.
Therefore, for h = 0, the partition function is given by:

\begin{equation}
Z = \lambda_{\text{max}}^N = (2 \cosh(\beta J))^N
\label{eq:partitionh0}
\end{equation}

The free energy \( F \) is given by:
\begin{equation}
F = -k_B T \ln(Z) = -k_B T N \ln(2 \cosh(\beta J))
\end{equation}

The internal energy \( U \) is given by:
\begin{equation}
U = - \frac{\partial}{\partial \beta} \ln(Z) = - \frac{\partial}{\partial \beta} N \ln(2 \cosh(\beta J)) = -JN \tanh(\beta J)
\end{equation}

The magnetization \( M \) is given by:
\begin{equation}
M = - \frac{\partial}{\partial h} \ln(Z) = - \frac{\partial}{\partial h} N \ln(2 \cosh(\beta J)) = 0
\end{equation}

The heat capacity \( C \) is given by:
\begin{equation}
C = - \beta^2 \frac{\partial^2}{\partial \beta^2} \ln(Z) = - \beta^2 \frac{\partial^2}{\partial \beta^2} N \ln(2 \cosh(\beta J)) = k_B \beta^2 N \text{sech}^2(\beta J)
\end{equation}

The susceptibility \( \chi \) is given by:
\begin{equation}
\chi = \beta \frac{\partial^2}{\partial h^2} \ln(Z) = \beta \frac{\partial^2}{\partial h^2} N \ln(2 \cosh(\beta J)) = 0
\end{equation}

These thermodynamic properties are fundamental to understanding the behavior of
the Ising Model. The heat capacity and susceptibility are both zero, indicating
that the system is stable and does not undergo a phase transition. This is
consistent with the physical behavior of the Ising Model, as the 1D Ising Model
does not exhibit a phase transition.
