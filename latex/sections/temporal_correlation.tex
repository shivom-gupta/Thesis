\section{Temporal Correlation}

When data is generated by using a Markov chain method, the data is temporally
correlated which can be seen in the autocorrelation function. The
autocorrelation function is defined as
\begin{equation}
  \label{eq:autocorrelation}
  A(k) = \frac{\langle \mathcal{O}_i \mathcal{O}_{i+k}\rangle - \langle\mathcal{O}_i\rangle\langle\mathcal{O}_i\rangle}{\langle\mathcal{O}_i^2\rangle- \langle\mathcal{O}_i\rangle\langle\mathcal{O}_i\rangle}
\end{equation}

where $\mathcal{O}_i$ is the observable at time $i$ for example the energy or
magnetization. The autocorrelation function is a measure of how correlated the
data is at a given time step. If the data is uncorrelated the autocorrelation
function will be zero for all $k$. For large time steps $k$ the autocorrelation
function $A(k)$ will decay exponentially as:

\begin{equation}
  \label{eq:autocorrelation_decay}
  A(k) \longrightarrow^{k\rightarrow\infty} A(0)e^{-k/\tau_{\mathcal{O}, exp}}
\end{equation}

where $\tau_{\mathcal{O}, exp}$ is the exponential autocorrelation time. The
autocorrelation function also contains some other modes because of which $A(k)$
does not decay exponentially for all $k$. 