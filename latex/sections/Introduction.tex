\section{Introduction}

The Ising model or  Lenz-Ising model, named after the german physicists Ernst
Ising and  Wilhelm Lenz, is a mathematical model for ferromagnetism in
statistical mechanics. It consists of discrete variables known as spins, which
can take on values of either +1 or -1. These spins are arranged on a lattice,
and the model defines the interactions between these spins.

The model consists of discrete variables, which represent the magnetic dipole
moments of atomic "spins". These spins, which can be in one of two states (+1 or
-1), are arranged in a lattice. This lattice structure repeats periodically in
all directions, allowing each spin to interact with its neighboring spins. The
key aspect of the model is that neighboring spins with the same orientation
(either both +1 or both -1) have lower energy than those with opposing
orientations. This setup allows the model to capture the fundamental behavior of
ferromagnetic materials.

One of the intriguing aspects of the Ising model is the way it balances energy
minimization and thermal disturbance. While the system naturally tends towards a
state of lowest energy, thermal fluctuations can disrupt this tendency, leading
to various structural phases. This dynamic is crucial for understanding phase
transitions in physical systems.

The concept of the Ising Model dates back to 1920 when Wilhelm Lenz proposed it,
later tackled by his student Ernst Ising. In his 1924 thesis, Ising solved the
one-dimensional version of the model, demonstrating the absence of a phase
transition in this case. The more complex two-dimensional square-lattice version
remained unsolved until 1944, when Lars Onsager provided a groundbreaking
analytic solution, marking a significant advancement in statistical mechanics.
