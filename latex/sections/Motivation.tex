\section{Motivation}
\label{sec:motivation}
\subsection{Why study The Ising Model?}
\textbf{1. Significance in Understanding Phase Transitions}

The Ising model is a quintessential tool in the study of phase transitions. It
demonstrates key phenomena such as:

\begin{itemize}
  \item \textit{Symmetry Breaking in Low-Temperature Phase}: As previously
  discussed, the Ising model showcases how symmetry is broken in the
  low-temperature phase, a fundamental concept in understanding phase
  transitions \cite{stanley1971phase,Binney2002}.
  \item \textit{Existence of a Critical Point}: The model features a distinct
  critical point at a well-defined temperature, analogous to the critical point
  in the phase diagram of water \cite{Feuerbacher1986}.
  \item \textit{Richness of Features}: Besides these, the Ising model harbors
  other rich features that deepen our understanding of phase transitions in
  various systems \cite{Behn1988}.
\end{itemize}

\textbf{2. Utility in Thermodynamics}

The Ising model stands out as one of the few exactly solvable models in
statistical mechanics. This is significant because:

\begin{itemize}
  \item \textit{Calculation of Thermodynamic Quantities}: Computing
  thermodynamic quantities in general involves summing over a large number of
  terms. Recall from introductory thermodynamics
  \cite{Tuckerman2023,Kardar2013,Yoshioka2007} that an equilibrium system can be
  viewed as an ensemble of many states \( s \), each with a probability \( P_s
  \). The observable thermodynamic quantities are averages over this ensemble.
  For an observable \( A(s) \), its ensemble average is \( \langle A \rangle =
  \sum_s A(s) P_s \). However, the challenge arises because the number of states
  scales exponentially with the number of particles in a system. For a system
  with \( N \) particles, where \( N \) is on the order of \( 10^{23} \), this
  becomes computationally infeasible.
  \item \textit{Importance of Exactly Solvable Systems}: Thus, the ability to
  exactly solve the Ising model \cite{Ising1925,Onsager1944} and compute its
  partition function is a significant achievement. It allows for precise
  calculations and predictions in a field where such precision is often
  unattainable.
\end{itemize}

\textbf{3. Universality and Applicability}

Lastly, the Ising model's simplicity belies its wide applicability:

\begin{itemize}
  \item \textit{First Encounter with Universality}: The Ising model introduces
  us to the concept of universality in critical phenomena. The same theoretical
  framework can describe a variety of different phase transitions, whether in
  liquids, gases, magnets, superconductors, or other systems.
  \item \textit{A Reflection of Deeper Order}: Such universal behavior is
  particularly intriguing to physicists as it suggests an underlying order in
  the seemingly chaotic natural world.
\end{itemize}

In conclusion, the Ising model is not just a theoretical construct but a vital
tool that offers profound insights into the complex world of phase transitions
and critical phenomena. Its simplicity, exact solvability, and universality make
it a cornerstone in the field of statistical mechanics and beyond.


\subsection{Why 1D Models?}
\begin{itemize}
  \item \textit{Testing Ground for Approximate Techniques}: One-dimensional (1D)
  models serve as a testing ground for the validity of approximate techniques
  developed for higher-dimensional systems.
  \item \textit{Insights into Higher-Dimensional Systems}: They help in guessing
  properties of higher-dimensional systems.
  \item \textit{Representation of Quasi 1D Systems}: Some materials behave as
  quasi one-dimensional systems \cite{Fava2020} \cite{Muller1993} due to
  dominating inter-particle interactions along a chain of particles.
  \item \textit{Mapping to 1D Ising-like Models}: There are classes of problems
  that can be mapped onto one-dimensional Ising-like models, such as the
  conformational equilibria of a linear polymer. \cite{Applequist1968}
  \item \textit{Applicability of Mathematical Techniques}: The mathematical
  techniques employed to solve one-dimensional problems are often applicable to
  other types of problems.
  \item \textit{Ordering at Absolute Zero}: Even the one-dimensional spin system
  must become ordered at \( T = 0 \). Therefore, understanding the behavior of
  the system as \( T \rightarrow 0 \) will be important in our treatment of spin
  systems via renormalization group methods.
\end{itemize}